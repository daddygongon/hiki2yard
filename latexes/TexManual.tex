\documentclass[10pt,a4j,twocolumn]{jsarticle}

\usepackage[dvipdfmx]{graphicx}

\setlength{\textheight}{275mm}
\headheight 5mm
\topmargin -30mm
\textwidth 185mm
\oddsidemargin -15mm
\evensidemargin -15mm
\pagestyle{empty}
\begin{document}
\section{TexManual}
\subsection{teXShopの設定}\begin{enumerate}
\item 『設定』 $\rightarrow$ ︎書類 $\rightarrow$ ︎エンコーディング $\rightarrow$ Unicode(UTF-8)
\item 『設定プロファイル』$\rightarrow$ pTeX(ptex2pdf)を選択
\item TeXShopを再起動
\end{enumerate}
\subsection{必要なgemの確認}\begin{itemize}
\item gem listコマンドを入力,以下のgemが入っているか確認\begin{itemize}
\item hikidoc
\item hikiutils
\end{itemize}
\item ︎︎無ければ(sudo) gem install hogehogeで入れる
\end{itemize}
\subsection{入力の方法}\begin{itemize}
\item open -a mi hogehogeでファイルを開き,入力していく
\end{itemize}
\subsection{pdfファイルへの変換方法}\begin{enumerate}
\item hiki2latex hikis/hogehoge.hiki $>$ latexes/hogehoge.tex
\end{enumerate}\begin{quote}\begin{verbatim}
 (注意).hikiはhikisフォルダに,.texファイルはlatexesフォルダに入れてください
\end{verbatim}\end{quote}\begin{enumerate}
\item open hogehoge.tex
\item TeXShopが開く. 左上のタイプセットを押すとpdfファイルが出力される
\end{enumerate}
\end{document}
