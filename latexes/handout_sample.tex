\documentclass[10pt,a4j,twocolumn]{jsarticle}

\usepackage[dvipdfmx]{graphicx}

\setlength{\textheight}{275mm}
\headheight 5mm
\topmargin -30mm
\textwidth 185mm
\oddsidemargin -15mm
\evensidemargin -15mm
\pagestyle{empty}
\begin{document}
\title{hiki2yardによる数値計算資料の作成システム}
\author{情報科学科 西谷研究室 1234 西谷滋人}
\date{}
\maketitle
\section{目的}
hikiフォーマット文書をlatexフォーマットに変換するhiki2latex[1]をもちいて中間発表のabstract資料を作成する手順を紹介する.

\section{インストール}
最初に,作業を自動化するhiki2yard[2]をrubygemsからinstallする.terminal上で,
\begin{quote}\begin{verbatim}
gem install hiki2yaml
\end{verbatim}\end{quote}
によって環境を構築するCUIがinstallされる.
\begin{quote}\begin{verbatim}
hiki2yard --version
\end{verbatim}\end{quote}
によってversionが表示されれば,正常にinstallがなされていることが確認できる.

\section{環境構築}
terminal上で
\begin{quote}\begin{verbatim}
hiki2yard --init
\end{verbatim}\end{quote}
によって,必要となるファイルが自動的に配置される.支持に従って,hogehoge.gemspecを
手動で修正する必要がある.さらに,
\begin{quote}\begin{verbatim}
bundle update
\end{verbatim}\end{quote}
によって,必要なgem filesがinstallされる.

\section{directory構成と各ファイルの意味}
次にdirectory構成を示す.これはbundle gem -bによって生成される標準gem構築環境に修正を加えた構成となる.
\begin{quote}\begin{verbatim}
bob% tree .
.
├── CODE_OF_CONDUCT.md
├── Gemfile
├── Gemfile.lock
├── LICENSE.txt
├── README.md
├── Rakefile
├── bin
├── doc
├── exe
├── hiki2yard.gemspec
├── hiki2yard.wiki
├── hikis
│   ├── README_en.hiki
│   ├── handout_sample.hiki
...
├── latexes
│   ├── handout_pre.tex
├── lib
├── pkg
└── spec
\end{verbatim}\end{quote}
hikisにtargetとなるfileをhiki構文で作る.Rakefileに必要なtaskが登録されている.latexesには現在のところ,中間発表のhandoutを生成するpre-formatのtexが置かれている.
\begin{quote}\begin{verbatim}
rake latex
\end{verbatim}\end{quote}
によって
\begin{quote}\begin{verbatim}
hiki2latex --pre latexes/handout_pre.tex hikis/handout_sample.hiki > latexes/handout_sample.tex
\end{verbatim}\end{quote}
が起動され,latexes上にtargetのtex format文書が生成される.この後,自動的に起動されるmacのapplicationであるTeXShop上でcommand-tを押下することで,latexからpdfへ変換を行い,中間発表のabstractを作成する.

\section{個別の準備}\begin{itemize}
\item 各人は自分でhikis/hogehoge.hikiファイルを生成する
\item Rakefileの:latexタスクにあるtarget変数を変更する
\end{itemize}
後は,マニュアルに従ってpdf文書を作成できるはずである.質問がある場合は早めにね.

\section{参考資料}\begin{enumerate}
\item \verb|hiki2latex(https://rubygems.org/gems/hiki2latex)|, 2016/08/11アクセス.
\item \verb|sakibts/hiki2yard(https://github.com/sakibts/hiki2yard)|, 2016/08/11アクセス.
\end{enumerate}
\end{document}
